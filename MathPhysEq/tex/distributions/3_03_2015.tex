\section{3.03.2015}
\subsection{Плотность множества основных функций $D(\Omega)$ в $L^2(\Omega)$}
\begin{lemma}
Пусть $\Omega$ -- ограниченное множество, тогда для любой функции $f \in L^2(\Omega)$ и для любого $\varepsilon > 0$ найдется такая функция $\phi \in D(\Omega)$, что выполняется неравенство:
$$
\norm{f - \phi}_{L^2} < \varepsilon
$$
\end{lemma}
\begin{proof}
   Comming soon.
\end{proof}

\subsection{Пространство обобщенных функций}
\emph{Обобщенной функцией} называется всякий линейный непрерывный функционал на пространстве основных функций $D$. Если функция $\phi \in D$, тогда множество функционалов от функции $\phi$ обозначим: 
$$
f(\phi) = \scalar{f}{\phi}
$$ 
где $f$ -- линейный непрерывный функционал. Он  обладает следующими свойствами:
\begin{enumerate}
 \item пусть $\phi_1, \phi_2 \in D$ и $\alpha_1, \alpha_2$ -- комплексные числа, тогда 
 $$
 \scalar{f}{\alpha_1 \phi_1 + \alpha_2 \phi_2} = \alpha_1 \scalar{f}{\phi_1} + \alpha_2 \scalar{f}{\phi_2};
 $$
 \item если $\phi_k \to \phi$  в  $D, k \to \infty,$ тогда $\scalar{f}{\phi_k} \to \scalar{f}{\phi}, k \to \infty$, в частности, если $\phi_k \to 0$  в  $D, k \to \infty,$ тогда $\scalar{f}{\phi_k} \to 0, k \to \infty$.
\end{enumerate}
Множество функционалов $f$ будем обозначать $D'.$

Часто линейный непрерывный функционал рассматривается в виде:
$$
\scalar{f}{\phi} = \integral{\fieldr^n}{}{f(x)\phi(x)}{x}.
$$
Линейность этого множества следует из линейности интеграла, а непрерывность для всех локально интегрируемых функций $f$ -- из возможности предельного перехода под знаком интеграла.
В общем случае пространство обобщенных функций $D'$ будем считать линейным, если линейную комбинацию $\alpha_1 f + \alpha_2 g$ обобщенных функций $f$ и $g$ определить как функционал, действующий по формуле:
$$
\scalar{\alpha_1 f + \alpha_2 g}{\phi} = \alpha_1 \scalar{f}{\phi} + \alpha_2 \scalar{g}{\phi}, \qquad \phi \in D.
$$

\begin{task}
	Доказать, что функционал $\alpha_1 f + \alpha_2 g$ линейный и непрерывный.
\end{task}


\begin{solution}
	Comming soon.
\end{solution} 


\subsection{Полнота пространства обобщенных функций}
\begin{lemma}
Пусть есть последовательность $\{f_k\}, f_k \in D',$ такая, что для каждой функции $\phi \in D$ числовая последовательность $\scalar{f_k}{\phi}$ сходиться при $k \to \infty,$ т. е. существует предел $\lim\limits_{k \to \infty} \scalar{f_k}{\phi}.$ Тогда функционал $f$ на $D$ определенный равенством:
$$
\scalar{f}{\phi} = \lim_{k \to \infty} \scalar{f_k}{\phi}, \qquad \phi \in D,
$$
также является линейным и непрерывным на $D,$ т. е. $f \in D'.$
\end{lemma}
\begin{proof}
   Comming soon.
\end{proof} 

\subsection{Носитель обобщенных функций}
Будем говорить, что обобщенная функция $f$ \emph{равна нулю в области} $\Omega,$ если для любой функции $\phi \in D(\Omega)$ справедливо равенство: $\scalar{f}{\phi} = 0.$
 
Две обобщенные функции $f$ и $g$ будем называть \emph{равными в области} $\Omega$  $(f = g),$ если для любой функции $\phi \in D(\Omega)$ справедливо равенство: $\scalar{f}{\phi} = \scalar{g}{\phi}.$

Пусть обобщенная функция $f$ равна нулю в области $\Omega.$ Тогда она равна равна нулю в любой подобласти области $\Omega$ и, следовательно, в окрестности любой точки области $\Omega.$ Справедливо и обратное.
\begin{lemma}
Если обобщенная функия $f$ равняется нулю в окрестности каждой точки области $\Omega,$ то она равна нулю во всей области $\Omega.$  
\end{lemma}
\begin{proof}
   См. Владимиров-Жаринов (2004), с. 72.
\end{proof}

\subsection{Регулярные обобщенные функции}
Функция $f$ называется \emph{локально интегрируемой в $\fieldr^n$,} если она интегрируема по любой компактной области в $\fieldr^n.$

Пусть $f \in L_1^{loc}(\fieldr^n),$ где $L_1^{loc}(\fieldr^n)$ -- множество локально интегрируемых функций в $\fieldr^n.$ Тогда функционал порождаемый функцией $f$ по формуле:
\begin{equation}
\scalar{f}{\phi} = \integral{}{}{f(x)\phi(x)}{x}, \qquad \phi \in D
\label{regularFunc}
\end{equation}
является обобщенной функцией.
 
Обобщенные функции, определяемыми локально интегрированными в $\fieldr^n$ функциями по формуле \eqref{regularFunc}, называются \emph{регулярными обобщенными функциями}.
\begin{lemma}[дю Буа-Реймона]
 Пусть обобщенная функция $f = 0$ в области $\Omega,$ тогда $f = 0$ почти всюду в $\Omega.$ 
\end{lemma}
\begin{proof}
   Сomming soon.
\end{proof}

\subsection{Сингулярные обобщенные функции}
Все обобщенные функции принадлежащие пространству $D'$ и не являющиеся регулярными, называются \emph{сингулярными.}

Основным примером сингулярной обобщенной функцией является $\delta$-функция Дирака.
Докажем что $\delta$-функция Дирака является обобщенной функцией.
Предположим противное, пусть существует локально интегрируемая в $\fieldr^n$ функция $f,$ такая что для любой функции $\phi \in D$
\begin{equation}
\scalar{f}{\phi} = \integral{}{}{f(x)\phi(x)}{x} = \phi(0).
\label{condition}
\end{equation}
Так как $x_1\phi(x) \in D,$ если $\phi \in D$ то из \eqref{condition} следует
$$
\integral{}{}{f(x)x_1\phi(x)}{x} = \substitute{x_1\phi(x)}{x = 0}{} = 0 = \scalar{x_1f}{\phi}
$$
при всех $\phi \in D;$ здесь $x_1$ -- первая координата $x.$ Таким образом, локально интегрируемая в $\fieldr^n$ функция $x_1f$ равна нулю в смысле обобщенных функций. По лемме дю Буа-Реймона $x_1f(x) = 0,$ а следовательно и $f(x) = 0$ почти всюду. Но это противоречит равенству \eqref{condition}. Полученное противоречие доказывает сингулярность $\delta$-функции.

\subsection{Обобщенные производные}
Пусть функция $f \in C^1$ и является локально интегрируемой в $\Omega, \phi \in D,$ тогда используя формулу интегрирования по частям
$$
 \integral{\Omega}{}{f'_{x_i}(x)g(x)}{x} = \integral{\partial \Omega}{}{f(x)g(x) \cos(\widehat{\vec \nu, \vec 0_{x_i}})}{\Gamma} - \integral{\Omega}{}{f(x)g'_{x_i}(x)}{x},
$$
получаем
\begin{align*}
	&\scalar{f}{\phi'_{x_i}} = \integral{\Omega}{}{f(x)\phi'_{x_i}(x)}{x} = \underbrace{\integral{\partial \Omega}{}{f(x)g(x) \cos(\widehat{\vec \nu, \vec 0_{x_i}})}{\Gamma}}_0 - \integral{\Omega}{}{f'(x)\phi(x)}{x} = \\
 &= - \integral{\Omega}{}{f'(x)\phi(x)}{x}.
\end{align*}

Пусть $\alpha = (\alpha_1, \ldots, \alpha_n)$ -- произвольный мультииндекс.
Пусть $f \in C^{|\alpha|},$ откуда следует, что $f$ -- регулярная обобщенная функция.
Тогда производную порядка $|\alpha|$ для любой регулярной функции можно определить по формуле:
$$
\scalar{f}{\partial^\alpha \phi} = (-1)^{|\alpha|} \scalar{\partial^\alpha f}{\phi}.
$$ 
Эта формула получается аналогично предыдущей, путем применения формулы интегрирования по частям $|\alpha|$ раз.

В качестве примера найдем производную функции одной переменной, имеющей разрыв первого рода. Обычно такие функции относятся к классу недифференцируемых.
\begin{example}
Пусть функция $f$ имеет разрыв первого рода в точке $x_0,$ а во всех остальных точках она является непрерывно дифференцируемой, функция $\phi$ такая, что $\phi(a) = \phi(b) = 0.$
Тогда
\begin{align*}
	\scalar{f'}{\phi} = -\integral{a}{b}{f(x)\phi'(x)}{x} = -\left( \integral{a}{x_0}{f(x)\phi'(x)}{x} + \integral{x_0}{b}{f(x)\phi'(x)}{x} \right).
\end{align*}
Применяя формулу интегрирования по частям к каждому их интегралов, получаем:
\begin{align*}
& \scalar{f'}{\phi} = -\substitute{f(x)\phi(x)}{a}{x_0-0} + \integral{a}{x_0}{f'(x)\phi(x)}{x} - \substitute{f(x)\phi(x)}{x_0+0}{b} + \\
& + \integral{x_0}{b}{f'(x)\phi(x)}{x} = -f(x_0 - 0)\phi(x_0) + \integral{a}{b}{\{f'(x)\}\phi(x)}{x} + \\ &+ f(x_0 +  0)\phi(x_0) = \phi(x_0)[f]_{x_0} + \integral{a}{b}{\{f'(x)\} \phi(x)}{x},	
\end{align*}
где $[f]_{x_0}$ -- скачок функции $f$ в точке $x_0, \{f'(x)\} $ -- производная функции $f$ в тех точках, где она существует.
Таким образом в смысле обобщенных функций получаем:
$$
 f' = [f]_{x_0}\delta(x - x_0) + \{f'(x)\},
$$  
где $\delta(x - x_0)$ -- $\delta$ -- функция Дирака, сосредоточенная в точке $x_0, \delta(x - x_0) =~ ~\phi(x_0).$
\end{example}

\begin{task}
	Найти производную от функции включения Хевисайда.
\end{task}


\begin{solution}
	Comming soon.
\end{solution} 

