\section{17.02.2015}
\subsection{$\delta$-функция Дирака}

Дирак ввел эту функцию для описания плотностей (масс, зарядов и др.) в столь малом объеме, что его можно принять за точку.

Исходя из того, что если $\delta(x)$ --- плотность распределения массы заряда $x = (x_1,x_2,x_3) \in \fieldr^3$.
\begin{align*}
    &\integral{}{}{\delta(x)}{x} = 1, \qquad \delta(x) = 0, \qquad x \ne 0; \\
    &\integral{\fieldr^3}{}{\delta(x)}{x} = \lim_{\epsilon \to 0}
    \integral{{\abs{x}>\epsilon}}{}{\delta(x)}{x}.
\end{align*}
Пусть $f(x)$ --- непрерывная в окрестности 0 функция. Рассмотрим
\begin{align*}
	&\delta_{\epsilon}(x) = 
		\begin{cases}
			&\frac{1}{2\epsilon}, \qquad \abs{x} > \epsilon, \\
			&0, \qquad \abs{x} \leq \epsilon;
		\end{cases} \qquad x \in \fieldr; \\
        &\integral{\fieldr}{}{\delta_{\epsilon}(x)f(x)}{x} = \lim_{\epsilon \to
        0} \integral{{\abs{x}>\epsilon}}{}{\delta_{\epsilon}(x)}{x} =
        \lim_{\epsilon \to 0} \integral{{\abs{x}>\epsilon}}{}{\frac{1}{2\epsilon}f(x)}{x}.
\end{align*}
Поскольку $\delta$ --- неотрицательная функция, можно воспользоваться I-ой теоремой о среднем и вынести значение в некоторой средней точке за знак интеграла.
$$
\lim_{\epsilon \to 0} \integral{{\abs{x}>\epsilon}}{}{\frac{1}{2\epsilon}f(x)}{x} = \lim_{\epsilon \to 0} {\frac{1}{2\epsilon} f(\xi_{\epsilon}) \integral{-\epsilon}{\epsilon}{}{x}} = \lim_{\epsilon \to 0} f(\xi_{\epsilon}) = f(0);
$$
Т.о. $\delta$-функция Дирака оказалась функционалом, который на каждой, непрерывной в окрестности 0 функции действует по правилу. 

Учитывая, что действие этого функционала прослеживается через предельный переход в интегральных операциях от $\delta$-образной последовательности, это действие записывается следующим образом
$$
\scalar{\delta}{\phi} = \phi(0); \qquad \integral{\fieldr^n}{}{\delta(x)\phi(x)}{x} = \phi(0).
$$

\begin{task}
	Привести примеры $\delta$-образных последовательностей в $\fieldr^2, \fieldr^3$. При этом использовать не только функции с разрывом 1-го рода, но и бесконечно дифференцируемые.
\end{task}

\begin{solution}
	Comming soon.
\end{solution}


\subsection{Пространство основных функций $D$}


$D = D(\fieldr^n)$ --- функции, имеющие конечный носитель в $\fieldr^n$, бесконечно дифференцируемые. В этом множестве вводится топология следующим образом. Последовательность функций $\phi_k \to \phi$ входит в $D$, если:
\begin{enumerate}
	\item $\exists R, \, \supp \phi_k \subset B_R$.
	\item $\alpha = (\alpha_1, \dots, \alpha_n)$ - мультииндекс. $D^{\alpha}\phi_k \rightrightarrows D^{\alpha}\phi$ в $B_R$. $D^{\alpha} = \frac{ \partial^{\abs{\alpha}} }{\partial x_1^{\alpha_1} \dots \, \partial x_n^{\alpha_n}}$, где $\abs{\alpha} = \sum\limits_{i=1}^n \alpha_i, \,\alpha_i \in \mathbb{Z}^+$.
\end{enumerate}


\begin{task}
	Доказать, что дифференциальные операторы непрерывны в топологии $D$.
\end{task}


\begin{solution}

	Comming soon.
\end{solution}


\begin{task}
	Доказать, что линейная замена переменных $y = Ax+b$ ($A$ --- невырожденная матрица и $b \in \fieldr^n$) --- непрерывная операция в топологии $D$.
\end{task}


\begin{solution}

	Comming soon.
\end{solution}


\begin{task}
	Доказать, что операция умножения на бесконечно дифференцируемую функцию непрерывна в $D$.
\end{task}


\begin{solution}

	Comming soon.
\end{solution}
