\section{17.02.2015}
\subsection{Понятие задачи Штурма-Лиувилля}

Рассмотрим линейное однородное дифференциальное уравнение второго порядка
\begin{equation}\label{eq:ode_second_order}
	Ly(x) = P_0(x) y''(x) + P_1(x) y'(x) + P_2(x) y(x) = 0,
\end{equation}
где $P_0(x) \ne 0 \text{ для } \forall x \in [a,b]$. Разделив уравнение на~$P_0(x)$, получаем
$$
	y''(x) + c(x)y'(x) + d(x)y(x) = 0.
$$
Будем рассматривать случай, когда $c(x) = c, d(x) = d$ --- константы. Найдем решение полученного уравнения. Для этого запишем характеристическое уравнение, затем выпишем общее решение.
\begin{align*}
	&k^2 + c k + d = 0; \\
	&y_{c}(x) =
	\begin{cases}
		&C_1 e^{k_1 x} + C_2 e^{k_2 x}, k = k_1 \ne k_2; \\
		&C_1 e^{k x} + C_2 x e^{k x}, k = k_1 = k_2; \\
		&e^{\alpha x} (C_1 \cos{\beta x} + C_2 \sin{\beta x}), k = \alpha \pm i \beta.
	\end{cases}
\end{align*}

Пусть $y_1(x)$ и $y_2(x)$ образуют фундаментальную систему решений. Тогда любое решение~$y(x)$ представимо в~виде
$$
	y(x) = C_1 y_1(x) + C_2 y_2(x).
$$
Вспомним, что решением ЛНДУ 2-го порядка $Ly(x) = f(x)$ является $y(x) = y_{c}(x) + y_{p}(x)$, где $y_{c}(x)$ --- общее решение однородного уравнения, а~$y_{p}(x)$~--- частное решение неоднородного уравнения.

\subsection{Двухточечная задача}
\begin{example}
	$$
		y''+ \lambda y = 0, \qquad y(0) = 0, \qquad y(l) = 0.
	$$
	Найдем решение данной задачи.
	\begin{align*}
		&k^2 + \lambda = 0; \\
		&k = \pm i \sqrt{\lambda}.
	\end{align*}
	Рассмотрим случай, когда $\lambda > 0$. Тогда общее решение будет иметь вид $y_{c}(x) = C_1 \cos{\sqrt{\lambda}\,x} + C_2 \sin{\sqrt{\lambda}\,x}$. Подставим граничные условия:
	\begin{align*}
		&y_{c}(0) = C_1 = 0 \Longrightarrow C_1 = 0; \\
		&y_{c}(l) = C_2 \sin{\sqrt{\lambda}\,l} = 0.
	\end{align*}
	Пусть $C_2 \ne 0$, тогда $l \sqrt\lambda = \pi n \text{, отсюда } \lambda = \frac{\pi^2 n^2}{l^2},\,n \in \mathbb{N}$.
\end{example}

\subsection{Понятие сопряженного дифференциального уравнения в $L^2(\Omega)$}
Пусть $\Omega = \set{x \in \fieldr }{a < x < b}$.
Будем рассматривать пространство
$$
L^2(\Omega) = \set{f}{(L)\integral{\Omega}{}{\abs{f(x)}^2}{x} < \infty, \, x \in \Omega, \, f \colon \Omega \to \fieldr}.
$$
Скалярное произведение и норма вводятся в этом пространстве следующим образом
\begin{align*}
    &\scalar{u}{v} = \integral{\Omega}{}{u(x) \widebar{v(x)}}{x},\qquad \forall u,v \in L^2(\Omega), \\
    &\norm{u}_{L^2} = \sqrt{\scalar{u}{u}} = \left( \integral{\Omega}{}{\abs{u(x)}^2}{x} \right)^{\frac{1}{2}}.
\end{align*}
Положим $H \subseteq L^2(\Omega)$. Задан оператор $A \colon H \to H$. $A^*$ --- сопряженный к~$A$ в~$H$, т.е. $\scalar{Au}{v} = \scalar{u}{A^*v}$. Возьмем $A = \fulld{x}$ и проверим, является ли он самосопряженным. Будем предполагать, что функции $u$ и/или $v$ имеют конечный носитель в $\Omega$.
\begin{align*}
    \scalar{\fulld{x}u}{v} &= \integral{\Omega}{}{\fulld{x}u(x) v(x)}{x} =
    \underbrace{\substitute{u(x)v(x)}{a}{b}}_0 - \integral{a}{b}{u(x) v'(x)}{x} = \\
    &= - \integral{a}{b}{u(x) \fulld{x}v(x)}{x} = \scalar{u}{-\fulld{x}v}.
\end{align*}
Таким образом получаем, что $A^* = -\fulld{x} \ne A$.
\begin{remark}
	Оператор $\frac{\mathrm{d}^2}{{\mathrm{d}x}^2}$ --- является самосопряженным оператором в $L^2(\Omega)$ при условии, что функция и её производная имеет конечный носитель на множестве интегрирования. Другой пример самосопряженного оператора --- умножение на бесконечно непрерывно-дифференцируемую функцию.
\end{remark}

Рассмотрим следующий дифференциальный оператор:
$$
	L = \fulld{x}\left[ \phi(x)\fulld{x} \right].
$$
Проверим, является ли он самосопряженным в $L^2(\Omega)$ (при условии, сказанном в замечании):
\begin{align*}
    \scalar{Lu}{v} &= \integral{\Omega}{}{\fulld{x}\left[ \phi(x)\fulld{x}u(x) \right] v(x)}{x} = \\
    &= \underbrace{\substitute{v(x)\phi(x)\fulld{x}u(x)}{a}{b}}_0 -
    \integral{a}{b}{\phi(x)\fulld{x}u(x) v'(x)}{x} = \\
&= - \underbrace{\substitute{u(x)\phi(x)\fulld{x}v(x)}{a}{b}}_0 +
\integral{a}{b}{u(x) \fulld{x}  \left[ \phi(x) \fulld{x}v(x) \right] }{x} = \scalar{u}{Lv}.
\end{align*}
Получаем, что $L$ --- самосопряженный оператор. В $L^2(\Omega)$ это общий вид самосопряженного оператора. Отвечающее ему уравнение записывается в виде
$$
	\fulld{x}\left[ \phi(x)\fulld{x}y \right] - q(x)y = 0.
$$

Рассмотрим общий способ приведения уравнения второго порядка к самосопряженному виду.
Домножим обе части уравнения \eqref{eq:ode_second_order} на функцию $\rho(x)$, которая не обращается в нуль:
\[ \rho(x) P_0(x) y'' + \rho(x) P_1(x) y' + \rho(x) P_2(x) y = 0. \]

Самосопряженное уравнение имеет вид
\[ \phi(x) y'' + \phi'(x) y' - q(x) y = 0. \]
Тогда, приравнивая множители при соответствующих производных функции $y$, получаем:
\begin{align*}
    \phi(x) &= \rho(x) P_0(x) \\
    \phi'(x) &= \rho(x) P_1(x) = \rho'(x) P_0(x) + \rho(x) P_0'(x).
\end{align*}
В результате имеем дифференциальное уравнение первого порядка относительно $\rho$:
\[ P_0(x) \rho'(x) = \rho(x) (P_1(x) - P_0'(x)). \]
Разделив переменные и проинтегрировав, получим
\begin{equation}\label{eq:rho_self_adjoint}
    \rho(x) = \frac{C}{P_0(x)} \exp\left\{\integral{}{}{\frac{P_1(x)}{P_0(x)}}{x}\right\}.
\end{equation}
