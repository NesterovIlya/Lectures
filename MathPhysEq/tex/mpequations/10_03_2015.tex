\section{10.03.2015}
\subsection{Постановка граничных задач для колебания струны(стержня)}

Будем рассматривать неоднородное волновое уравнение колебаний струны
\begin{equation}\label{eq:inhom_wave}
	u_{tt}'' = a^2 u_{xx}'' + f(x,t)
\end{equation}
с начальными условиями
\begin{equation}\label{eq:inhom_wave_init_cond}
	\left\{
        \begin{aligned}
			u(x,0) &= \phi_1(x), \\
			u_t'(x,0) &= \phi_2(x);
		\end{aligned}
    \right.
\end{equation}
где $f(x,t)$ -- функция внешнего воздействия, $u(x,0)$ -- положение точек струны в~начальный момент времени, 
$u_t'(x,0)$ -- начальные скорости точек струны.

Граничные условия могут быть следующих видов:
\begin{enumerate}
	\item I рода:

	\begin{equation}\label{eq:inhom_wave_bound_cond1}
		u(0,t) = \psi_1(t), \qquad u(l,t) = \psi_2(t); 
	\end{equation}
	где $\psi_1(t)$ и $\psi_2(t)$ -- уравнения движения концов в~процессе колебаний.

	\item II рода:

	\begin{equation}\label{eq:inhom_wave_bound_cond2}
		u_x'(0,t) = \nu_1(t), \qquad u_x'(l,t) = \nu_2(t), 
	\end{equation}
	Поскольку по \emph{закону Гука} натяжение пропорционально деформации, а деформация в~безразмерном виде представляется в~виде производной, то производная показывает
	с~точностью до~некоторой константы внешнее усилие, которое подчиняется закону~$\nu_i(t)$.

	\item III рода:

	\begin{equation}\label{eq:inhom_wave_bound_cond3}
		u_x'(l,t) - \alpha u(l,t) = \mu(t),
	\end{equation}
	где $\alpha$ -- жесткость закрепленного конца струны. Данное уравнение определяет условие упругого закрепления концов струны. Упругое закрепление
	означает, что возникающее усилие вызывает обратную реакцию, которая пропорциональна смещению.
\end{enumerate}

\begin{task}
	Каким будет условие свободных концов?
\end{task}

\begin{solution}
	Comming soon.
\end{solution}

\begin{task}
	Как записать условие жесткого крепления концов?
\end{task}

\begin{solution}
	Comming soon.
\end{solution}

\subsection{Вывод уравнения теплопроводности}

Пусть имеется произвольный объем $V \subset \fieldr^3$. Обозначим через $u(x,t)$ температуру в~каждой точке объема, $x = (x_1,x_2,x_3) \in V$.

Вспомним следующие определения из курса математического анализа.
\begin{definition}
    Пусть функция $u$ дифференцируема в точке $\\(x_0,y_0,z_0)$. Тогда в~этой точке функция $u$ имеет \emph{производную по направлению} $l \in \fieldr^3$ и эта производная
    находится по формуле
    \begin{equation}\label{eq:direct_deriv}
		\pd{u}{l} = \scalar{\nabla u}{\ort l},
	\end{equation}
	где $\nabla{u} = \left( \pd{u}{x_1},\pd{u}{x_2},\pd{u}{x_3} \right)$ -- градиент функции $u$, $\ort{l} = \frac{l}{\abs{l}}$ -- направляющие косинусы вектора направления.
\end{definition} 

Так как $\ort{l}$ -- единичный вектор, то $\abs{\ort{l}} = 1$, поэтому уравнение (\ref{eq:direct_deriv}) запишется в виде
\[ \pd{u}{l} = \abs{\nabla{u}} \cos\phi, \]
где $\phi$ -- угол, образованный вектором $l$ и $\nabla{u}$. Если в данной точке $\abs{\nabla{u}}^2 \ne 0$, то производная по~направлению достигает наибольшего значения в~единственном 
направлении, а~именно том, при~котором $\cos\phi = 1$, т.е. в~направлении градиента.

\begin{definition}
	Пусть задано векторное поле $a = (a_x, a_y, a_z)$ в некоторой области $G$, дифференцируемое в некоторой точке.
	Число $\\ \pd{a_x}{x} + \pd{a_y}{y} + \pd{a_z}{z} $ называется \emph{дивергенцией} поля в этой точке и обозначается через $\divg{a}$, т.е.
	\[ \divg{a} = \pd{a_x}{x} + \pd{a_y}{y} + \pd{a_z}{z} = \nabla{a}. \]
\end{definition}

Вернемся к исходной задаче. Известно, что градиент температур некоторого объема создает тепловой поток. Обозначим через $S$ границу объема $V$, и пусть $\nu$ -- внешняя нормаль к~ней.
Согласно закону Фурье через поверхность $S$ в~объем $V$ за~промежуток времени $(t,t+\Delta t)$ поступает количество тепла
\begin{equation}\label{eq:dQ_across_bound1}
	\Delta Q = - \iintegral{S}{}{ k \scalar{\nabla u}{\nu} }{S}\Delta t - \iiintegral{V}{}{ f(x,t) }{x}\Delta t,
\end{equation}
где $k$ -- коэффициент теплопроводности, $\dd S$ -- элемент поверхности, $f(x,t)$ -- плотность внутренних источников тепла.
Воспользуемся формулой Гаусса-Остроградского и приведем (\ref{eq:dQ_across_bound1}) к виду
\begin{equation}\label{eq:dQ_across_bound2}
	\Delta Q = -\iiintegral{V}{}{ \Bigl[  \divg{(k \nabla u)} + f(x,t) \Bigr] }{x}\Delta t.
\end{equation}
Посмотрим с точки зрения эмпирической формулы Ньютона:
\begin{equation}\label{eq:dQ_across_bound_newton}
	\Delta Q = \gamma m \left( u(x,t+\Delta t) - u(x,t) \right) = - \iiintegral{V}{}{ \gamma \rho u_t'(x,t) }{x}\Delta t.
\end{equation}
Согласно закону сохранения количества тепла, значения (\ref{eq:dQ_across_bound2}) и (\ref{eq:dQ_across_bound_newton}) должны совпадать.
$$
	\iiintegral{V}{}{ \Bigl[ \gamma \rho u_t'(x,t) - \divg{(k \nabla u)} - f(x,t) \Bigr] }{x} = 0.
$$
Будем считать, что все производные в~этом выражении есть непрерывные функции, тогда ввиду произвольности области $V$ подинтегральная функция равна 0.
Получаем 
\begin{equation}\label{eq:heat1}
	u_t'(x,t) = \frac{1}{\gamma\rho}\divg{(k \nabla u)} + \frac{1}{\gamma\rho}f(x,t).
\end{equation}
Если среда однородна, т.е $k,\rho,\gamma$ -- постоянные, то уравнение (\ref{eq:heat1}) принимает вид
\begin{equation}\label{eq:heat2}
	u_t'(x,t) = a^2 \Delta u + F, \qquad a^2 = \frac{k}{\gamma\rho}, \qquad \frac{f}{\gamma\rho},
\end{equation}
где $\Delta = \nabla^2$ -- оператор Лапласа.
Уравнение (\ref{eq:heat2}) называется \emph{уравнением теплопроводности}.

\subsection{Постановка краевых задач для уравнения теплопроводности}

Рассмотрим возможные начальные и граничные условия для уравнения теплопроводности (\ref{eq:heat2}).

Начальные условия:
\begin{equation}\label{eq:heat_init_cond}
	u(x,0) = \phi(x),
\end{equation}
где $\phi(x)$ -- температура точек тела в начальный момент времени.

Граничные условия:
\begin{enumerate}
	\item (I рода) Если на границе $S$ поддерживается заданное распределение температуры $\psi_1(x,t)$, где $x \in S$, то
	\begin{equation}\label{eq:heat_bound_cond1}
		\left. u
		\right|_S^{} = \psi_1(x,t);
	\end{equation}

	\item (II рода) Если на границе $S$ поддерживается заданный поток тепла $\psi_2(x,t)$, где $x \in S$, то
	\begin{equation}\label{eq:heat_bound_cond2}
		\left. -k\pd{u}{\nu}
		\right|_S^{} = \psi_2(x,t);
	\end{equation}

	\item (III рода) Если на границе $S$ происходит теплообмен согласно закону Ньютона, то
	\begin{equation}\label{eq:heat_bound_cond3}
		\left. \Bigl[ k\pd{u}{\nu} + h(u-u_0) \Bigr]
		\right|_S^{} = 0,
	\end{equation}
	где $h$ -- коэффициент теплообмена, $u_0$ -- температура окружающей среды
\end{enumerate}

Если в уравнении теплопроводности посчитать, что в~процессе теплообмена функция стабилизируется и становится независимой от времени, то получим уравнение
\begin{equation}\label{eq:poisson}
	a^2\Delta u + F = 0.
\end{equation}
Уравнение (\ref{eq:poisson}) называется \emph{уравнением Пуассона}, а при $F = 0$ \emph{уравнением Лапласа}.

Основные краевые задачи:
\begin{enumerate}
	\item (I рода) Задача Дирихле.
	\[  
		\left. u
		\right|_S^{} = \psi_1(x);
	\]

	\item (II рода) Задача Неймана.
	\[  
		\left. \pd{u}{\nu} = \psi_2(x)
		\right|_S^{} = \psi_2(x);
	\]

	\item (III рода) 
	\[  
		\left. \Bigl[ k\pd{u}{\nu} + hu \Bigr]
		\right|_S^{} = \psi_3(x,u,z);
	\]
\end{enumerate}