\section{24.02.2015}
\subsection{Приведение уравнения Бесселя к самосопряженному виду}
Рассмотрим \emph{уравнение Бесселя}, имеющее вид
\[ x^2 y'' + xy' + (x^2 - p^2)y = 0. \]

Подставляя коэффициенты уравнения в выражение \ref{eq:rho_self_adjoint}, получаем
\[ \rho(x) = \frac{1}{x^2} e^{\integral{}{}{\frac{1}{x}}{x}} = \frac{1}{x}. \]

Тогда поделим уравнение Бесселя на $x$:
\[ xy'' + y' + \left(x - \frac{p^2}{x}\right) y = 0, \]
или иначе
\[ (xy')' + \left(x - \frac{p^2}{x}\right) y = 0. \]
Это \emph{уравнение Бесселя в самосопряженной форме}.

Расмотрим \emph{сингулярный оператор Бесселя}:
\[B_\gamma = \frac{1}{x^\gamma} \fulld{x}{\left[x^\gamma \fulld{x}\right]}, \quad \gamma > 0. \]

Такой оператор является самосопряженным в пространстве $L_\gamma^2(\Omega)$ --- квадратично-суммируемых с весом $\gamma$ функций:
\[ L_\gamma^2 = \set{f \colon \Omega \to \fieldr}{\integral{\Omega}{}{\absv{f(x)}^2 x^\gamma}{x} < \infty}, \]
скалярное произведение в котором определяется равенством
\[ \scalar{u}{v}_\gamma = \integral{\Omega}{}{u(x)v(x) x^\gamma}{x}. \]

В самом деле, пусть $u, v \in C_0^2(\Omega)$ --- дважды непрерывно дифференцируемые функции с конечным носителем. Тогда
\begin{equation*}
\begin{split}
    \scalar{B_\gamma u}{v}_\gamma &= \integral{}{}{\frac{1}{x^\gamma} \fulld{x}\left[x^\gamma \frac{\mathrm{d}u}{\mathrm{d}x}\right] v(x) x^\gamma}{x} = -\integral{}{}{x^\gamma \frac{\mathrm{d}u}{\mathrm{d}x} \frac{\mathrm{d}v}{\mathrm{d}x}}{x} \\
    &= \integral{}{}{\frac{1}{x^\gamma} \fulld{x}\left[x^\gamma \frac{\mathrm{d}v}{\mathrm{d}x}\right] u(x) x^\gamma}{x} = \scalar{u}{B_\gamma v}_\gamma.
\end{split}
\end{equation*}
