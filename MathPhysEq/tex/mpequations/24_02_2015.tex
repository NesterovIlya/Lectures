\section{24.02.2015}
\subsection{Приведение уравнения Бесселя к самосопряженному виду}
Рассмотрим \emph{уравнение Бесселя}, имеющее вид
\[ x^2 y'' + xy' + (x^2 - p^2)y = 0. \]

Подставляя коэффициенты уравнения в выражение \ref{eq:rho_self_adjoint}, получаем
\[ \rho(x) = \frac{1}{x^2} e^{\integral{}{}{\frac{1}{x}}{x}} = \frac{1}{x}. \]

Тогда поделим уравнение Бесселя на $x$:
\[ xy'' + y' + \left(x - \frac{p^2}{x}\right) y = 0, \]
или иначе
\[ (xy')' + \left(x - \frac{p^2}{x}\right) y = 0. \]
Это \emph{уравнение Бесселя в самосопряженной форме}.

Расмотрим \emph{сингулярный оператор Бесселя}:
\[B_\gamma = \frac{1}{x^\gamma} \fulld{x}{\left[x^\gamma \fulld{x}\right]}, \quad \gamma > 0. \]

Такой оператор является самосопряженным в пространстве $L_\gamma^2(\Omega)$ --- квадратично-суммируемых с весом $\gamma$ функций:
\[ L_\gamma^2 = \set{f \colon \Omega \to \fieldr}{\integral{\Omega}{}{\absv{f(x)}^2 x^\gamma}{x} < \infty}, \]
скалярное произведение в котором определяется равенством
\[ \scalar{u}{v}_\gamma = \integral{\Omega}{}{u(x)v(x) x^\gamma}{x}. \]

В самом деле, пусть $u, v \in C_0^2(\Omega)$ --- дважды непрерывно дифференцируемые функции с конечным носителем. Тогда
\begin{equation*}
\begin{split}
    \scalar{B_\gamma u}{v}_\gamma &= \integral{}{}{\frac{1}{x^\gamma} \fulld{x}\left[x^\gamma \frac{\mathrm{d}u}{\mathrm{d}x}\right] v(x) x^\gamma}{x} = -\integral{}{}{x^\gamma \frac{\mathrm{d}u}{\mathrm{d}x} \frac{\mathrm{d}v}{\mathrm{d}x}}{x} \\
    &= \integral{}{}{\frac{1}{x^\gamma} \fulld{x}\left[x^\gamma \frac{\mathrm{d}v}{\mathrm{d}x}\right] u(x) x^\gamma}{x} = \scalar{u}{B_\gamma v}_\gamma.
\end{split}
\end{equation*}

\subsection{Собственные числа и собственные функции задачи Штурма-Лиувилля}
Рассмотрим уравнение Штурма-Лиувилля
\begin{equation}\label{eq:sturm_liouville}
    [\phi(x)y']' - q(x) y + \lambda \rho(x) y = 0
\end{equation}
с граничными условиями вида
\begin{equation}\label{eq:s_l_boundary}
\left\{
\begin{aligned}
    \alpha_1 y(a) + \alpha_2 y'(a) &= 0, \\
    \beta_1 y(b) + \beta_2 y'(b) &= 0,
\end{aligned}
\right.
\end{equation}
где $\absv{\alpha_1} + \absv{\alpha_2} \neq 0$ и $\absv{\alpha_1} + \absv{\alpha_2} \neq 0$, а функции $\phi$ и $\rho$ положительны на отрезке $[a, b]$.

Уравнение \eqref{eq:sturm_liouville} вместе с граничными условиями \eqref{eq:s_l_boundary} называются \emph{задачей Штурма-Лиувилля} (Ш.-Л.).

Значения $\lambda \in \fieldr$, для которых задача Ш.-Л. имеет ненулевое решение, называются \emph{собственными числами задачи Ш.-Л.} Сами же ненулевые решения --- \emph{собственными функциями задачи Ш.-Л., соответствующими собственному числу $\lambda$}.

\begin{theorem}
    Пусть $u_1$ и $u_2$ --- собственные функции, соответствующие собственному числу $\lambda$. Тогда они линейно зависимы, т. е. $u_1 = c u_2$, $c \neq 0$.
\end{theorem}

\begin{proof}
    Предположим противное: пусть $u_1$ и $u_2$ линейно независимы. Тогда, поскольку они оба удовлетворяют ЛДУ II порядка, их определитель Вронского не обращается в нуль ни в одной точке (см. курс ОДУ):
    \[
        W(u_1(x), u_2(x)) = \begin{vmatrix}
            u_1(x) & u_2(x) \\
            u_1'(x) & u_2'(x)
        \end{vmatrix} \neq 0.
    \]
    Но в точке $x = a$, в соответствии с условиями \eqref{eq:s_l_boundary} получаем
    \[ \left\{
        \begin{aligned}
            \alpha_1 u_1(a) + \alpha_2 u_1'(a) &= 0, \\
            \alpha_1 u_2(a) + \alpha_2 u_2'(a) &= 0. \\
        \end{aligned}
        \right.
    \]
    Поскольку $\alpha_1$ и $\alpha_2$ одновременно не обращаются в нуль, получаем, что система имеет ненулевое решение относительно переменных $\alpha_1$ и $\alpha_2$, а значит её определитель равен нулю:
    \[
        \begin{vmatrix}
            u_1(a) & u_2(a) \\
            u_1'(a) & u_2'(a)
        \end{vmatrix} = 0.
    \]
    Получили противоречие.
\end{proof}

\begin{definition}
    Функции $u$ и $v$ называются \emph{ортогональными с весом $\rho$} на отрезке $[a, b]$, если
    \[ \integral{a}{b}{u(x) v(x) \rho(x)}{x} = 0. \]
\end{definition}

\begin{theorem}
    Пусть $u_1$ и $u_2$ --- собственные функции задачи Ш.-Л. (\ref{eq:sturm_liouville}, \ref{eq:s_l_boundary}), отвечающие различным собственным числам $\lambda_1$ и $\lambda_2$ соответственно. Тогда они ортогональны с весом $\rho$ на отрезке $[a, b]$.
\end{theorem}

\begin{proof}
    Поскольку $u_1$ и $u_2$ решения, имеем:
    \begin{align*}
        [\phi u_1']' - q u_1 + \lambda_1 \rho(x) u_1 &= 0, \\
        [\phi u_2']' - q u_2 + \lambda_2 \rho(x) u_2 &= 0.
    \end{align*}
    Домножим первое уравнение на $u_2$, а второе на $u_1$:
    \begin{align*}
        u_2 [\phi u_1']' - q u_1 u_2 + \lambda_1 \rho(x) u_1 u_2 &= 0, \\
        u_1 [\phi u_2']' - q u_1 u_2 + \lambda_2 \rho(x) u_1 u_2 &= 0.
    \end{align*}
    Вычитая первое из второго, получаем:
    \[ u_1 [\phi u_2']' - u_2 [\phi u_1']' = (\lambda_1 - \lambda_2) u_1 u_2 \rho(x). \]
    Левая часть этого равенства преобразуется к виду
    \[ u_1 [\phi u_2']' - u_2 [\phi u_1']' = [\phi (u_1 u_2' - u_2 u_1')]' = [\phi W(u_1(x), u_2(x))]', \]
    (это проверяется непосредственно). Тогда
    \[ (\lambda_1 - \lambda_2) u_1 u_2 \rho(x) = [\phi W(u_1(x), u_2(x))]'. \]
    Проинтегрируем обе части равенства по отрезку $[a, b]$ и используем формулу Ньютона-Лейбница:
    \begin{align*}
        (\lambda_1 &- \lambda_2) \integral{a}{b}{u_1(x) u_2(x) \rho(x)}{x} =
        \integral{a}{b}{[\phi W(u_1(x), u_2(x))]'}{x} = \\ &=
        \substitute{\phi W(u_1(x), u_2(x))}{a}{b} = \phi(b) W(u_1(b), u_2(b)) -
        \phi(a) W(u_1(a), u_2(a)) = 0,
    \end{align*}
    где $W(u_1(a), u_2(a)) = W(u_1(b), u_2(b)) = 0$ в силу граничных условий \eqref{eq:s_l_boundary} (аналогично предыдущей теореме).
\end{proof}
